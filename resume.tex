\documentclass[12pt,a4paper]{article}
\usepackage[T1]{fontenc}
\usepackage[a4paper]{geometry}
\usepackage[portuges,brazilian]{babel}
\usepackage[utf8]{inputenc}
\usepackage{setspace}
\usepackage{libertine}
\usepackage{graphicx}	
\usepackage{ragged2e} 
\usepackage{authblk}
\usepackage{hyperref}		
%
\newcommand*{\email}[1]{%
    \normalsize\href{mailto:#1}{#1}\par
    }


\begin{document} 
\begin{figure}
    \flushright
    \includegraphics[scale=0.5]{Logo_senai.png}
\end{figure}

\title{Roteiro \\ Robôs de Segurança de patrimônio} 
\author{Anderson Queiroz}
\affil{Senai Cimatec. CCRoSA - Centro de Competência em Robótica e Sitemas Autônomos \\ \email{anderson.vale@fbter.org.br}}

 

    \maketitle
    \pagenumbering{arabic}
    \singlespacing


    \section{Público Alvo}
    Pesquisadores, Professores, alunos, empresários e curiosos da tecnologia que queiram saber mais sobre robôs de segurança, o que há de novo e os impactos desse tema na sociedade. Estimular a pesquisa, o desenvolvimento e financiamento de projetos relacionados a robôs de segurança de patrimônio.

    \textbf{Tempo: 20-30 minutos}

    \section{Perfil}
    Perfil de uma pessoa séria, preocupada e melhoria da segurança do país e ao mesmo tempo passar uma imagem amigável, sem passar uma visão assustadora sobre o tema. Afinal, assuntos sobre violência e mortes precisa ser discutido com seriedade.

    \section{Contextualização do Roteiro}
    \subsection{Introduzir tema central}
    Será discutido o assunto de Robôs de segurança de patrimônio. Neste tema será mostrado as vantagens e desvantagens desses equipamentos na sociedade, o que já existe atualmente no cenário mundial e o que está por vim. Falar também um pouco sobre os limites.

    \subsection{Introduzir o contexto histórico da problemática}
    Historicamente, a violência de modo geral sempre foi um dos principais causadores de traumas nas pessoas. Um país inseguro transmite uma população insegura e com medo. Passar por um momento de extrema violência pode deixar marcas eternas em uma pessoa. Mesmo aqueles países que estão no topo como os mais seguros, como Islândia, Nova Zelândia e Portugal, ainda atingem números significativo de homicídios, assaltos e furtos.

    \subsection{Contar um pouco a minha história em Portugal sobre a segurança}
    Estar seguro traz uma sensação de liberdade em que você pode ir e vim para onde quiser sem se preocupar muito se irá voltar pra casa. Ainda lembro minha mãe e minha vó sempre dizendo todas as vezes que eu ia para a escola: “Vai com Deus e volte com Ele em segurança.”. Antes de ir para Portugal meus familiares sempre se preocuparam com minha segurança e me orientaram a ser bastante cauteloso e atento a tudo, sendo que eu estava indo sozinho para um lugar desconhecido. Demorou um pouco de tempo para perceber a diferença de segurança entre o Brasil e Portugal. Eu tive essa percepção quando estava em um bairro dito perigoso pelos portugueses, totalmente deserto, escuro e sozinho e nesse lugar avistei uma mulher com o celular na mão passando ao meu lado. Certamente aqui no Brasil jamais isso aconteceria e muito provavelmente as chances de assaltos seriam muito maiores. 

    Depois disso o medo e a insegurança sumiram inexplicavelmente. Sempre andei e continuei muito atento e cauteloso mas as preocupações de segurança desapareceram e o sentimento de liberdade é incrível, ir para onde quiser sem se preocupar com os bloqueios no caminho.

    \section{A falta de segurança nas ruas, no comercio e nas residências}
    Aqui no Brasil nem mesmo em nossas casas temos a sensação completa de segurança. Sempre temos que trancar nossas portas e janelas muito bem. Se deixarmos aberto ou esquecermos de trancar pode acontecer o pior. Minha casa já foi invadida várias vezes mas graças aos meu cachorro espantava os invasores. Quando o meu “cão de guarda” morreu deixou uma sensação de insegurança e as noites não era mais tranquilas até ter outro. Mesmo o cachorro que apenas espantava os invasores já passava uma segurança maior no ambiente. 

    O mesmo acontece nos comércios. Você precisa contratar um segurança particular e colocar câmeras mas nem isso intimida os ladrões. Isso porque o segurança particular demora para ter uma reação e chamar a polícia sem que os ladrões vejam e os comerciantes podem estar sendo feito de refém e também não consegue entrar em contato com a polícia.

    Outro problema pode ser apresentado nas ruas. Não existe policial de patrulhamento para todos os cantos e mesmo aqueles seguranças de bairros não conseguem dar conta vigiar todas as casas e lugares. 

    \section{O conflito: Apresentar a diferença entre os robôs de segurança e os guardas de segurança}
    Desde então, câmeras de vigilância e guardas de segurança sempre foram os que mantiveram as pessoas e estabelecimentos seguros e protegidos, mas pode haver alguns inconvenientes para eles: 

    \textbf{Custo} (equipamento, salário) - A contratação de guardas de segurança às vezes podem ser caros. Além de seus salários, instalar câmeras de vigilância que podem cobrir tudo e todos os cantos também podem ser caro.

    \textbf{Atraso} - Os guardas de segurança tendem a se atrasar para o trabalho, que as vezes pode ser um fardo/aborrecimento para os guardas que precisam cobrir eles.

    \textbf{Preguiça} - Eles tendem a relaxar em seus empregos e realizam o trabalho sem entusiasmo.

    \textbf{Suborno} - O dinheiro é poderoso, especialmente estes dias, por isso não é surpresa se algum segurança oficiais aceitarem os subornos para que possam estar fora do gancho.

    \textbf{Vigilância} - Mesmo se eles estão assistindo, ainda há chances deles perderam algumas ações suspeitas.

    \textbf{Riscos} - Pode haver casos em que as vidas dos agentes de segurança podem ser colocadas em risco e isso deve ser evitado.

    Esses são os vários problemas que nossa segurança precisa ser sustentado e ainda mais porque essas questões podem custar as pessoas muito dinheiro e tempo. Os robôs de segurança pode ser a solução:

    \textbf{Custo} - Como mencionado antes, contratar seguranças pode ser caro. No entanto, alugar um robô de segurança e contratar poucos guardas são muito mais baratos que contratar muitos funcionários de segurança para apenas um estabelecimento.

    \textbf{Atraso} - Os robôs de segurança nunca se atrasam para seus trabalhos. Eles podem trabalhar 24h por dia, desde que eles tenham a capacidade de carregar autonomamente.

    \textbf{Preguiça} - Os robôs fazem o que são programados para fazer. Eles não param de fazer o trabalho deles até você desligá-los.

    \textbf{Suborno} - Os robôs não param totalmente o suborno, como ainda há funcionários humanos que estão envolvidos na execução estratégica do trabalho do robô, porém eles poderiam gravá-lo e minimizar o ato de suborno

    \textbf{Vigilância} - Os robôs de segurança tem câmeras que captura os arredores em 360 graus e que certamente supera a instalação de várias câmeras de vigilância em uma área.

    \textbf{Riscos} - Os robôs podem impedir que humanos arrisquem suas vidas sempre que houver uma situação de perigo.

    \section{Apresentar os robôs de segurança e sua história}
    Agora imaginem ter os robôs de patrulha que pode transitar pelas ruas e comércios monitorando e identificando eventuais problemas de segurança a todo momento.  (Neste momento pode ser colocar um vídeo introdutório de um robô patrulhando).

    Atualmente a pesquisa sobre os robôs de segurança estão crescendo e será uma tendência para o futuro. Existe vários tipos de robôs para a segurança

    \section{Mostra a diferença entre os robôs de segurança, patrulha, policiais, militar e combate.}
    Para vocês não se confundirem e achar que um robô de segurança é cheio de armas e que podem atirar em pessoas, não é! (Mostrar aqui as imagens e vídeos dos robôs, apresentando as diferenças entre eles)

    \section{Limites dos robôs de segurança (Regulamentação)}
    Apresentar os desafios éticos enfrentados pelo desenvolvimento de sistemas robóticos que empregam força violenta e letal contra seres humanos. Embora o uso de força violenta e letal não seja geralmente aceitável para humanos ou robôs, os policiais são autorizados pelo estado a usar violência e força letal em certas circunstâncias, a fim de manter a paz e proteger indivíduos e a comunidade de uma ameaça imediata. Com o crescente interesse em desenvolver e implantação de robôs para tarefas policiais, incluindo robôs armados, a questão surge se, ou como, projetar interações com robôs humanos nas quais forças violentas e letais pode estar entre as ações executadas pelo robô. Isso representa um "problema mortal de design". Embora seja possível projetar um sistema para reconhecer vários gestos, como “Mãos ao alto, não atire! ”, há muitos aspectos mais desafiadores e sutis no problema de implementar diretrizes legais existentes para o uso da força em robôs de aplicação da lei. Os principais desafios jurídicos e técnicos da criação de interações envolvendo violência, algumas reflexões sobre a ética do projeto de HRI (Human-Robot Interaction) levantadas automatizando ouso da força no policiamento.  

    Três Leis da Robótica são, na verdade, três regras e/ou princípios idealizados pelo escritor Isaac Asimov a fim de permitir o controle e limitar os comportamentos dos robôs que este trazia à existência em seus livros de ficção científica.

    1ª Lei: Um robô não pode ferir um ser humano ou, por inação, permitir que um ser humano sofra algum mal.

    2ª Lei: Um robô deve obedecer as ordens que lhe sejam dadas por seres humanos exceto nos casos em que tais ordens entrem em conflito com a Primeira Lei.

    3ª Lei: Um robô deve proteger sua própria existência desde que tal proteção não entre em conflito com a Primeira ou Segunda Leis.

    Mais tarde Asimov acrescentou a “Lei Zero”, acima de todas as outras: um robô não pode causar mal à humanidade ou, por omissão, permitir que a humanidade sofra algum mal

    \section{Concluindo o tema}
    Apesar dos robôs estarem, atualmente, infringindo as leis de Asimov, a humanidade possui essa tendência de ter mais robôs automatizados na questão da preservação e segurança do patrimônio e das pessoas.

    Sabemos que no Brasil muitos policiais e seguranças são mortos por causa da violência. Os robôs de segurança pode ajudar a reduzir esse número e trazer mais segurança para as cidades com suas patrulhas.  

\end{document}